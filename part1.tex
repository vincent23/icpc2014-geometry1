\begin{frame}{Floating Point Beispiel}
	\lstset{
		language=C++,
		tabsize=2
	}
	\lstinputlisting{error.cpp}
	Ausgabe?
\end{frame}

\begin{frame}{Unerwartete Ergebnisse}
	\begin{exampleblock}{Ausgabe}
		3.5999999046325683594 \\
		false\\
		false
	\end{exampleblock}
	3.6 nicht darstellbar \\
	Rundungsfehler \\
	Seltsame Sonderwerte:
	\begin{itemize}
		\item NaN
		\item Inf
		\item +-0
	\end{itemize}
\end{frame}

\begin{frame}{IEEE 754 Knigge}
	\begin{itemize}
		\item Floats \& Doubles möglichst vermeiden
		\item Wenn nötig, erst ganz spät von Ganzzahl zu Fließkomma wechseln
		\item Bitte KEIN float und NUR double benutzen
		\item Keine direkten Vergleiche
	\end{itemize}
\end{frame}

\begin{frame}{Punkte}
	\begin{block}{Rotation um Ursprung (2D)}
		$
		\begin{bmatrix}
			x'\\
			y'
		\end{bmatrix}
		=
		\begin{bmatrix}
			\cos(\theta) & -\sin(\theta)\\
			\sin(\theta) & \cos(\theta)
		\end{bmatrix}
		\times
		\begin{bmatrix}
			x\\
			y
		\end{bmatrix}
		$
	\end{block}
	\begin{itemize}
		\item Rotationen um Punkte mittels Translationen
		\item In 3D um Achsen $\Rightarrow$ Zusätzliche Identitätszeile
	\end{itemize}
	
	Schreibt man sich einmal auf und sucht es sich bei Gelegenheit wieder raus
\end{frame}
